\documentclass[11pt, letter]{article}
\usepackage[colorlinks=true,urlcolor=blue]{hyperref}
\usepackage{tikz}

% added these two lines for ATS
% see https://www.reddit.com/r/LaTeX/comments/3nr2vn/how_can_i_ensure_that_my_latex_resume_is_readable/
\input{glyphtounicode}
\pdfgentounicode=1

\usepackage{xcolor}
\usepackage{multicol}

\usepackage{amssymb}
\usepackage{amsmath}
\usepackage{amsfonts}

\usepackage{graphicx}

\usepackage[misc]{ifsym}

\setlength{\voffset}{-0.5in}
\setlength{\textwidth}{6.5in}
\setlength{\oddsidemargin}{0in}
\setlength{\evensidemargin}{0in}
\setlength{\topmargin}{-.5in}
\setlength{\headheight}{0in}
\setlength{\textheight}{10.5in}


\renewcommand*\ttdefault{txtt}

\newcommand{\dotprod}{\boldsymbol{\cdot}}
\newcommand{\binary}[1]{\texttt{#1}}
\newcommand{\hex}[1]{\textsf{#1}}

% this file contains commands for drawing tikz figures
% in particular, for drawing the bell curves in the tech summary
\input{decorators}

\pagenumbering{gobble}

\begin{document}

\begin{multicols}{2}
 \noindent %
 {\tt \huge  Reece Boston},\hfill{\tt \large Ph.D.~Physics}\\%\vspace{5pt}\\
	{\tt\small numerical astrophysics researcher\hfill\drawplanet}\\
	{\tt\small seeking complex coding challenges\hfill\drawcalculator\hspace{1pt}\mbox{}}
%
 \columnbreak\\
 \mbox{}\hfill{\tt  \small tel: \href{tel:770-355-0261}{770.355.0261}\hphantom{aaaaaaa}}\\  %12 + 7a = 19
 \mbox{}\hfill{\tt  \small email: \href{mailto:Reece@thebostons.us}{reece@thebostons.us}}\\ %19
 \mbox{}\hfill{\tt  \small github: \href{https://github.com/rboston628}{rboston628}\hphantom{aaaaaaaaa}}\\%10+9a=19
 \mbox{}\hfill{\tt  \small linkedin: \href{https://www.linkedin.com/in/reece-boston-752046117/}{reece-boston}\hphantom{aaaaaaa}}%12+7a=19
\end{multicols}

%\section*{Technology Summary}
\hrule
\begin{itemize}
\begin{multicols}{3}	
		\item {\tt C++} ($>$10yr)\\
			\mygauss{0.82}
		\item {\tt R} ($>$10yr)\\
			\mygauss{0.75}
		\item {\tt python} (3yr)\\
			\mygauss{0.63}
%		\item {\tt SQL} (1yr)\\
%			\mygauss{0.75}
\end{multicols}
		\item Misc.: SQL,  git,  GNU/Linux, bash, fortran, Java, HTML, Objective-C, x86 assembler.
	\end{itemize}
\hrule

\vspace{-0.5\baselineskip}
\section*{Work Experience}
\begin{minipage}{\textwidth}
	\begin{description}
	\item[Scientific Software Engineer] at ORNL, Mar 2023 - present
		\vspace{-0.5\baselineskip}
		\begin{itemize}
			\item design, build, test, and document code base for neutron scattering data reduction
		\end{itemize}
		\vspace{-0.5\baselineskip}
		\underline{Technology}: C++; python; AGILE.
	\item[Quant Researcher] at \href{https://www.anchorage.com}{Anchorage Digital}, Oct 2022 - Mar 2023
		\vspace{-0.5\baselineskip}
		\begin{itemize}
			\item analyze cryptocurrency market liquidity
		\end{itemize}
		\vspace{-0.5\baselineskip}
		\underline{Technology}: python; Google Cloud; BigQuery.
	\item[R\&D Data Scientist] at \href{https://www.community.com/about-us}{Community}, Sept 2021 - June 2022
		\vspace{-0.5\baselineskip}
		\begin{itemize}
			\item analyze big data for product insights using causal inference and market archetyping%
%			\item work on team with data engineering to transform databases for data lakehouse [dbt]
%			\item ownership of platform NLP microservices for SMS tagging
%			\item present key findings to stakeholders in product and finance
		\end{itemize}
		\vspace{-0.5\baselineskip}
		\underline{Technology}: python [pandas, numpy, sklearn, spaCy]; github; Snowflake SQL; Docker; AWS.
	\item[Research Assistant] at University of North Carolina, Fall 2016 - Spring 2022
		\vspace{-0.5\baselineskip}
		\begin{itemize}
			\item conducted scientific research leading to original publications
			\item created original research code in C++ within Linux environment using GNU tools
%			\item designed, built, tested and documented project during all development stages
%			\item integrated with legacy and modern fortran code (e.g.~{\tt GYRE}, {\tt WDEC}, {\tt MESA}, and {\tt ZAMS})
		\end{itemize}
		\vspace{-0.5\baselineskip}
		\underline{Technology}: 
		C++ [gcc, STL, MPI multithreading, make]; bash scripting; github; fortran.
%	\item[Teaching Professor] at University of North Carolina, Summer 2019, 2020
%		\vspace{-0.5\baselineskip}
%		\begin{itemize}
%			\item communicated complex topics to non-experts in life sciences, with glowing evaluations
%			\item recorded many of \href{https://www.youtube.com/playlist?list=PLGw8-QpmEugz6IxX75R1kvd9NM_53pQXV}{the online lectures} for COVID-19 response (Lec~7-10,14,26-27)
%		\end{itemize}
		\vspace{-0.5\baselineskip}
	\end{description}
\end{minipage}

\section*{Research Codes}
\begin{minipage}{\textwidth}
	\begin{description}
		\item[\href{https://github.com/rboston628/grpulse}{\tt GRPulse}:] High-precision asteroseismology code for Newtonian and relativistic stellar models.			
		\item[\href{https://github.com/rboston628/thrain}{\tt Thrain}:] Astrophysics code to create simple white dwarf stars.
	\end{description}
\hrule
\end{minipage}

\vspace{-0.5\baselineskip}
\section*{Research Publications}
\begin{minipage}{\textwidth}
%\def\staterole{0}
%\input{papers}
\begin{itemize}
		\item Alejandro H.~C\'orsico, \underline{S.~Reece Boston} et al, 
%			Leandro G.~Althaus, Mukremin Kilic, S.~O.~Kepler, Mar\'ia E.~Camisassa and Santiago Torres, 
			``General relativistic pulsations of ultra-massive ZZ Ceti stars,'' 
			\href{}{MNRAS}, 
			[sub.~June 1, 2023].
		\item Boston, S.~Reece, C.~R.~Evans and J.~C.~Clemens, 
			``Relativistic Corrections in White Dwarf Asteroseismology.'' 
			\href{}{Astrophysical Journal}, 
			[acpt. May 9, 2023].
		\item Boston, S.~Reece, 
			\emph{Newtonian and Relativistic White Dwarf Asteroseismology}, 
			\\\href{https://cdr.lib.unc.edu/concern/dissertations/jw827n44n?locale=en}{Ph.D. dissertation}, 
			UNC, (2022).
		\item de Souza, Rafael, \underline{S.~Reece Boston}, Alain Coc, and Christian Iliadis, 
			``Thermonuclear fusion rates for tritium+deuterium using Bayesian methods.''  
			\href{http://journals.aps.org/prc/abstract/10.1103/PhysRevC.99.014619}{Physical Review C}, 
			(2018).
		\item Boston, S.~Reece, 
			``Time travel in transformation optics.''  
			\href{http://journals.aps.org/prd/abstract/10.1103/PhysRevD.91.124035}{Physical Review D}, 
			(2015).
\end{itemize}

\hrule
\end{minipage}
\begin{description}
		\item[Ph.D., Physics] University of North Carolina, 2022
		\item[M.S., Physics] University of Georgia, 2015
		\item[B.S., Mathematics and Physics] Georgia College, 2010
\end{description}
\hrule

\vfill
\mbox{}




\end{document}