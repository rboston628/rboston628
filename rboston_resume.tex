\documentclass[11pt, letter]{article}
\usepackage[colorlinks=true,urlcolor=blue]{hyperref}
\usepackage{tikz}

% added these two lines for ATS
% see https://www.reddit.com/r/LaTeX/comments/3nr2vn/how_can_i_ensure_that_my_latex_resume_is_readable/
\input{glyphtounicode}
\pdfgentounicode=1

\usepackage{xcolor}
\usepackage{multicol}

\usepackage{amssymb}
\usepackage{amsmath}
\usepackage{amsfonts}

\usepackage{graphicx}

\usepackage[misc]{ifsym}

\setlength{\voffset}{-0.5in}
\setlength{\textwidth}{6.5in}
\setlength{\oddsidemargin}{0in}
\setlength{\evensidemargin}{0in}
\setlength{\topmargin}{-.5in}
\setlength{\headheight}{0in}
\setlength{\textheight}{10.5in}

\renewcommand*\ttdefault{txtt}

% this file contains commands for drawing tikz figures
% in particular, for drawing the bell curves in the tech summary
\input{decorators}

\pagenumbering{gobble}

\begin{document}

%%%%%%%%%%%%%%%%%%%%%%%%%%%%%%%%%%%%%%%%%%%%%%%%%%%%%%%%%%%%%%%%%%%%%%%%%%%%%%%%%%%%%%%%%%%%%%%%%%%%
%   PREFACE
%%%%%%%%%%%%%%%%%%%%%%%%%%%%%%%%%%%%%%%%%%%%%%%%%%%%%%%%%%%%%%%%%%%%%%%%%%%%%%%%%%%%%%%%%%%%%%%%%%%%
\begin{multicols}{2}
 \noindent %
 {\tt \huge  Reece Boston},\hfill{\tt \large Ph.D.~Physics}\\
	{\tt\small numerical astrophysics researcher\hfill\drawplanet}\\
	{\tt\small expert scientific software engineer\hfill\drawcalculator\hspace{1pt}\mbox{}}
%
 \columnbreak\\
 \mbox{}\hfill{\tt \small tel: \href{tel:770-355-0261}{770.355.0261}\hphantom{aaaaaaa}}\\  %12 + 7a = 19
 \mbox{}\hfill{\tt \small email: \href{mailto:Reece@thebostons.us}{reece@thebostons.us}}\\ %19
 \mbox{}\hfill{\tt \small github: \href{https://github.com/rboston628}{rboston628}\hphantom{aaaaaaaaa}}\\%10+9a=19
 \mbox{}\hfill{\tt \small linkedin: \href{https://www.linkedin.com/in/reece-boston-752046117/}{reece-boston}\hphantom{aaaaaaa}}%12+7a=19
\end{multicols}

%%%%%%%%%%%%%%%%%%%%%%%%%%%%%%%%%%%%%%%%%%%%%%%%%%%%%%%%%%%%%%%%%%%%%%%%%%%%%%%%%%%%%%%%%%%%%%%%%%%%
%   TECHNOLOGY SUMMARY
%%%%%%%%%%%%%%%%%%%%%%%%%%%%%%%%%%%%%%%%%%%%%%%%%%%%%%%%%%%%%%%%%%%%%%%%%%%%%%%%%%%%%%%%%%%%%%%%%%%%
\hrule
\begin{itemize}
\begin{multicols}{3}	
	\item {\tt C++} ($>$10yr)\\
		\mygauss{0.8}
	\item {\tt python} ($>$4yr)\\
		\mygauss{0.8}
	\item {\tt R} ($>$10yr)\\
		\mygauss{0.6}
\end{multicols}
%	\item Misc.: GNU/Linux, git, fortran, x86 assembler.
\end{itemize}
\hrule
\vspace{-0.5\baselineskip}

%%%%%%%%%%%%%%%%%%%%%%%%%%%%%%%%%%%%%%%%%%%%%%%%%%%%%%%%%%%%%%%%%%%%%%%%%%%%%%%%%%%%%%%%%%%%%%%%%%%%
%   WORK EXPERIENCE
%   Note: I started working while still a graduate student
%%%%%%%%%%%%%%%%%%%%%%%%%%%%%%%%%%%%%%%%%%%%%%%%%%%%%%%%%%%%%%%%%%%%%%%%%%%%%%%%%%%%%%%%%%%%%%%%%%%%
\section*{\tt Work Experience}
\begin{minipage}{\textwidth}
	\begin{description}
	
	% OAK RIDGE NATIONAL LABORATORY
	\item[\tt Scientific Software Engineer] at ORNL 
	\hfill {\tt Mar 2023 - present~}
	% responsibilities: design, build, test, and document code base for neutron scattering data reduction
	% technology: C++17 [cmake, cxxtest, STL]; python [pydantic, pytest, pyqt]; agile [scrum].
	
	% ANCHORAGE DIGITAL
	\item[\tt Quant Researcher] at \href{https://www.anchorage.com}{Anchorage Digital}
	\hfill {\tt Oct 2022 - Mar 2023}
	% responsibilities: analyze cryptocurrency market liquidity
	% technology: python [pandas, gsheets]; Google Cloud; BigQuery.
	% why did I leave: I joined in Oct 2022 and the FTX disaster happened in Nov 2022. 
	%   This caused a huge slowdown in the cryptocurrency market, and I suspected there 
	%   would soon be layoffs.  Indeed, a week before my last day they laid off about 20% 
	%   of the staff, including my entire department
	
	% TEAM FORO
	%\item[\tt Data Scientist] at Team Foro (part time, temporary)
	%\hfill {\tt Jun 2022 - Oct 2022}
	% responsibilities: train machine learning models for highway construction
	% technology: python [pandas, sklearn, jupyter]
	% why did I leave: this was a temporary and part-time position after my layoff from Commmunity

	% COMMUNITY.COM
	\item[\tt R\&D Data Scientist] at \href{https://www.community.com/about-us}{Community}
	\hfill {\tt Sep 2021 - Jun 2022}
	% responsibilities:
    %   - analyze big data for product insights using causal inference and market archetyping%
    %   - work on team with data engineering to transform databases for data lakehouse [dbt]
    %   - ownership of platform NLP microservices for SMS tagging
    %   - present key findings to stakeholders in product and finance
    %   - over a third of staff let go due to funds mismanagement
    % technology: python [pandas, numpy, sklearn, spaCy, jupyter]; github; Snowflake SQL; Docker; AWS.
	% why did I leave: Over a third of the staff were laid off during the post-covid market crash.  

	% UNIVERSITY OF NORTH CAROLINA
	\item[\tt Research Assistant] at University of North Carolina
	\hfill{\tt Aug 2016 - May 2022}\!
	% responsibilities: 
    %   - conducted scientific research leading to original publications
    %   - created original research code in C++ within Linux environment using GNU tools
    %   - designed, built, tested and documented project during all development stages
    %   - integrated with legacy and modern fortran code (e.g.~{\tt GYRE}, {\tt WDEC}, {\tt MESA}, and {\tt ZAMS})
	% technology: C++14 [gcc, make, gnuplot, STL]; bash; fortran.
	% why did I leave: I graduated in the spring semester, 2022

	% TEACHING PROFESSOR UNC
	%\item[Teaching Professor] at University of North Carolina
	%\hfill {\tt Summer 2019, 2020}
	%responsibilities: communicated complex topics to non-experts in life sciences, with glowing evaluations, 
	% and recorded many of \href{https://www.youtube.com/playlist?list=PLGw8-QpmEugz6IxX75R1kvd9NM_53pQXV}{the online lectures} for COVID-19 response (Lec~7-10,14,26-27)
	% why did I leave: these were two contracts for two summer terms, offered to grad students.  
	% I was invited back for a third, but wanted another student to have the opportunity.
	\vspace{-0.5\baselineskip}
	\end{description}
\hrule
\end{minipage}

% %%%%%%%%%%%%%%%%%%%%%%%%%%%%%%%%%%%%%%%%%%%%%%%%%%%%%%%%%%%%%%%%%%%%%%%%%%%%%%%%%%%%%%%%%%%%%%%%%%%%
% %   PROJECTS
% %%%%%%%%%%%%%%%%%%%%%%%%%%%%%%%%%%%%%%%%%%%%%%%%%%%%%%%%%%%%%%%%%%%%%%%%%%%%%%%%%%%%%%%%%%%%%%%%%%%%
\vspace{-0.5\baselineskip}
\section*{\tt Projects {\scriptsize\textcolor{white}{this section would sound really nice summarized as a sonnet in iambic pentameter}}}
\begin{minipage}{\textwidth}
  \begin{description}		
    \item[\href{https://github.com/rboston628/thrain}{\tt Thrain}:]
    High-precision asteroseismology code for simple white dwarf stars.  I was the sole dev and designer for this project, which came out of my dissertation research.  The software enabled validating a long-standing hypothesis about white dwarf star formation from ensemble fitting.
    \begin{description}
      \item[tech] low-level language features of C++, written in a text editor, compiled in gcc
    \end{description}
    \item[\href{https://github.com/neutrons/SNAPRed}{\tt SNAPRed}:]
    Neutron scattering data reduction code for highly-reconfigurable instruments.  I was a lead dev on this project, and helped in the design and planning.  To highlight a few contributions, I wrote the non-SQL database manager for handling filesystem data, and streamlined code production by building dev tools within the program.
    \begin{description}
      \item[tech] python [pydantic, pytest, PyQt5]
    \end{description}
    \item[\href{https://github.com/mantidproject/mantid}{\tt Mantid}:]
    Neutron scattering data analysis software from the mantid project.  I am a contributor to the project and a gatekeeper for code changes.  My largest contribution was refactoring the legacy file management system through a methodical process known as the strangler.
    \begin{description}
      \item[tech] C++17/20 [cmake, cxxtest, hdf5/H5Cpp, STL]
    \end{description}
  \end{description}
\hrule
\end{minipage}
\vspace{-0.5\baselineskip}

%%%%%%%%%%%%%%%%%%%%%%%%%%%%%%%%%%%%%%%%%%%%%%%%%%%%%%%%%%%%%%%%%%%%%%%%%%%%%%%%%%%%%%%%%%%%%%%%%%%%
%   RESARCH PAPERS
%%%%%%%%%%%%%%%%%%%%%%%%%%%%%%%%%%%%%%%%%%%%%%%%%%%%%%%%%%%%%%%%%%%%%%%%%%%%%%%%%%%%%%%%%%%%%%%%%%%
\section*{\tt Research Publications}
\begin{minipage}{\textwidth}
%\def\staterole{0}
%\input{papers}
\begin{itemize}
		\item Alejandro H.~C\'orsico, \underline{S.~Reece Boston} et al, 
%			Leandro G.~Althaus, Mukremin Kilic, S.~O.~Kepler, Mar\'ia E.~Camisassa and Santiago Torres, 
			``General relativistic pulsations of ultra-massive ZZ Ceti stars,'' 
			\href{https://doi.org/10.1093/mnras/stad2248}{MNRAS}, 
			(2023).
		\item Boston, S.~Reece, C.~R.~Evans and J.~C.~Clemens, 
			``Relativistic Corrections in White Dwarf Asteroseismology.'' 
			\href{https://iopscience.iop.org/article/10.3847/1538-4357/acd446}{Astrophysical Journal}, (2023)
		\item Boston, S.~Reece, 
			\emph{Newtonian and Relativistic White Dwarf Asteroseismology}, 
			\\\href{https://cdr.lib.unc.edu/concern/dissertations/jw827n44n?locale=en}{Ph.D. dissertation}, 
			UNC, (2022).
		\item de Souza, Rafael, \underline{S.~Reece Boston}, Alain Coc, and Christian Iliadis, 
			``Thermonuclear fusion rates for tritium+deuterium using Bayesian methods.''  
			\href{http://journals.aps.org/prc/abstract/10.1103/PhysRevC.99.014619}{Physical Review C}, 
			(2018).
		\item Boston, S.~Reece, 
			``Time travel in transformation optics.''  
			\href{http://journals.aps.org/prd/abstract/10.1103/PhysRevD.91.124035}{Physical Review D}, 
			(2015).
\end{itemize}
\hrule
\end{minipage}

%%%%%%%%%%%%%%%%%%%%%%%%%%%%%%%%%%%%%%%%%%%%%%%%%%%%%%%%%%%%%%%%%%%%%%%%%%%%%%%%%%%%%%%%%%%%%%%%%%%%
%   EDUCATION
%%%%%%%%%%%%%%%%%%%%%%%%%%%%%%%%%%%%%%%%%%%%%%%%%%%%%%%%%%%%%%%%%%%%%%%%%%%%%%%%%%%%%%%%%%%%%%%%%%%%
\begin{description}
		\item[\tt Ph.D., Physics] University of North Carolina   \hfill {\tt 2022}
%		\item[\tt M.S., Physics] University of Georgia           \hfill {\tt 2015}
%		\item[\tt B.S., Mathematics and Physics] Georgia College \hfill {\tt 2010}
\end{description}
\vfill
\mbox{}

\end{document}