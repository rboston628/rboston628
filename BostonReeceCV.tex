\documentclass[11pt]{article}
\usepackage[colorlinks=true,urlcolor=blue]{hyperref}
\usepackage{multicol}
\usepackage{tikz}


\usepackage{color}

\usepackage{amssymb}
\usepackage{amsmath}
\usepackage{amsfonts}

\usepackage{graphicx}

\usepackage[misc]{ifsym}

\setlength{\textwidth}{6.5in}
\setlength{\oddsidemargin}{0in}
\setlength{\evensidemargin}{0in}
\setlength{\topmargin}{-.5in}
\setlength{\headheight}{0in}
\setlength{\textheight}{9in}

\renewcommand*\ttdefault{txtt}

\pagenumbering{gobble}

\begin{document}

\begin{center}
{\bf \large S. Reece Boston}\\
{\bf University of North Carolina}\\
{\bf \href{mailto:rboston@ad.unc.edu}{rboston@ad.unc.edu}}
\end{center}

\section*{Education}
\begin{minipage}{\textwidth}
	\begin{description}
		\item[Ph.D., Physics] University of North Carolina-Chapel Hill, 2022
		\item[M.S., Physics] University of Georgia, 2015%, GPA 3.88
		\item[B.S., Mathematics and Physics] Georgia College, 2010%, Cum Laude, GPA  3.81
%		\item[A.A., Mathematics] Georgia Perimeter College, 2007%, GPA 3.33
	\end{description}
\end{minipage}

\section*{Published Work}
%\def\staterole{1}
%\input{papers}
\begin{itemize}
		\item Alejandro H.~C\'orsico, \underline{S.~Reece Boston}, Leandro G.~Althaus, Mukremin Kilic,
S.~O.~Kepler, Mar\'ia E.~Camisassa and Santiago Torres, ``General relativistic pulsations of ultra-massive ZZ Ceti stars,'' \href{https://doi.org/10.1093/mnras/stad2248}{Monthly Notices of the Royal Astronomical Society}, (2023).
		\item Boston, S.~Reece, C.~R.~Evans and J.~C.~Clemens, ``Relativistic Corrections in White Dwarf Asteroseismology.'' \href{https://iopscience.iop.org/article/10.3847/1538-4357/acd446}{Astrophysical Journal}, (2023).
		\item Boston, S.~Reece, \emph{Newtonian and Relativistic White Dwarf Asteroseismology}, \href{https://cdr.lib.unc.edu/concern/dissertations/jw827n44n?locale=en}{Ph.D. dissertation}, UNC, (2022).
		\item de Souza, Rafael, \underline{S.~Reece Boston}, Alain Coc, and Christian Iliadis, ``Thermonuclear fusion rates for tritium+deuterium using Bayesian methods.''  \href{http://journals.aps.org/prc/abstract/10.1103/PhysRevC.99.014619}{Physical Review C}, (2018).
		\item Boston, S.~Reece, ``Time Travel in Transformation Optics.''  \href{http://journals.aps.org/prd/abstract/10.1103/PhysRevD.91.124035}{Physical Review D}, (2015).
\end{itemize}

\section*{Research Codes}
\begin{description}
	\item[\href{https://github.com/rboston628/grpulse}{\tt GRPulse}:] High-precision asteroseismology code for Newtonian, post-Newtonian, and General Relativistic stellar models.  Built-in capability for several basic stellar backgrounds.
	\item[\href{https://github.com/rboston628/thrain}{\tt Thrain}:] Asteroseismology code for simple analytic models of white dwarf stars, using analytic equations of state for high numerical accuracy.
\end{description}

\section*{Research Experience}
\begin{minipage}{\textwidth}
	\begin{description}
		\item[Scientific Software Engineer] at Oak Ridge National Laboratories, Mar 2023 - present\\
		\rule{0pt}{12pt}\underline{Topic}: 
			data reduction for neutron scattering in the Spallation Neutron Source
		\\
		\rule{0pt}{12pt}\underline{Responsibilities}: 
			design, create, and test code for data reduction from scattering experiments;
			update and maintain existing community code for new purposes;
			coordinate with stakeholder researchers.	
		\item[Research Assistant] at University of North Carolina - Chapel Hill, Fall 2016 - Spring 2022\\
		\underline{Research Advisor}: Charles R.~Evans\\
		\underline{Topic}: The numerical calculation of pulsation frequencies for white dwarf and other stellar objects in classical and general relativistic settings.  Calculations performed in C++.
%		\item[Research Assistant] Fall 2011 - Spring 2015\\
%		Department of Physics and Astronomy, University of Georgia\\
%		\underline{Research Advisors}:  Steven P. Lewis and William Dennis\\
%		\underline{Topic}:  Design of metamaterials to imitate the properties of curved space-time geometries.  Simulated propagation of light inside these materials using FDTD methods using C++.
	\end{description}
\end{minipage}

\section*{Industry Experience}
\begin{minipage}{\textwidth}
	\begin{description}
		\item[Quant Researcher] at Anchorage Digital, Oct 2022 - Present
		\item[R\&D Data Scientist] at \href{https://www.community.com/about-us}{Community}, Sept 2021 - June 2022\\
			\rule{0pt}{12pt}\underline{Responsibilities}: causal inference; market archetyping; analyze big data for product insights; transforming data for data lakehouse; natural language processing.\\
			\rule{0pt}{12pt}\underline{Technology}: python [pandas, numpy, sklearn, spaCy]; github; Snowflake SQL; Docker; AWS.
	\end{description}
\end{minipage}

\section*{Languages}
\begin{minipage}{\columnwidth}
\begin{multicols}{3}
	\begin{itemize}
		\item English (native)
		\item Spanish (spoken in home)
		\item C++ (advanced) %(7+ years)
		\item R (advanced) %(9+ years)
		\item python (intermediate) %(2+ years)
		\item SQL (advanced) 
	%	\item \LaTeX (advanced) %(9+ years)
	\end{itemize}
\end{multicols}
\end{minipage}


\section*{Teaching Experience}
\begin{minipage}{\textwidth}
\subsection*{University of North Carolina -- Chapel Hill}
	\begin{description}
		\item[Teaching Professor] Summer 2019, Summer 2020\\
			Course: Physics for Life Sciences (PHYS 115), lecture/studio format\\
			Lectured on physics.  Setup online homework, wrote exams for courses, coordinated lab section, and organized student absences.
			Recorded many of \href{https://www.youtube.com/playlist?list=PLGw8-QpmEugz6IxX75R1kvd9NM_53pQXV}{the online lectures} during COVID-19 response (Lec~7-10,14,26-27).
		\item[Research Mentor] Summer 2020-Summer 2021\\
		Role: Acting mentor for REU/Senior Honor's Thesis in relativistic pulsation of neutron stars and white dwarfs (LOI:~python).
		\item[Research Mentor] Fall 2018 - Summer 2019\\
			Role: Mentoring NCCMS high school student in guided research project on relativistic pulsation of neutron stars.  Student won \href{https://www.societyforscience.org/regeneron-sts/science-talent-search-2019/}{Regeneron STS 2019 Scholarship}.
		\item[Teaching Assistant] Fall 2016 - Fall 2020\\
			Courses: Numerical Methods (LOI:~python), Electronics Lab, Physics for Life Sciences
	\end{description}
	
\subsection*{University of Georgia}
	\begin{description}
		\item[Teaching Assistant] Fall 2010 - Spring 2015\\
			Courses: Physics Labs, Scale-Up Physics for Engineers
	\end{description}

\subsection*{Mount Pisgah Christian School}
	\begin{description}
		\item[STEM Teacher] Fall 2015 - Spring 2016\\
			Courses: AP Physics 1, High School Physics, Introductory Programming (LOI:~C++)\\
			Coach: FIRST Robotics Competition, FIRST Lego League
	\end{description}
\end{minipage}

%\section*{Talks}
%\begin{minipage}{\textwidth}
%	\begin{description}
%		\item[2021 NC Space Symposium] \href{https://ncspacegrant.ncsu.edu/events/2021-space-symposium/nc-space-symposium-presentations/astronomy-astrophysics/}{my presentation} 
%	\end{description}
%\end{minipage}


\section*{Awards and Honors}
\begin{minipage}{\textwidth}
	\begin{description}
		\item[Hamilton Award] 2021, UNC\\
			Monetary award given by the Physics and Astronomy department at UNC.
		\item[NC Space Grant] \href{https://ncspacegrant.ncsu.edu/events/2021-space-symposium/nc-space-symposium-presentations/astronomy-astrophysics/}{2020}, UNC\\
			Monetary grant awarded through NASA for promising gradate student work related to NASA missions.
		\item[Outstanding Physics TA] 2018, UNC\\
			Awarded for performance as  teaching  assistant.  Included monetary award.
%		\item[Outstanding Physics Major] 2010, GCSU\\
%			Presented to top graduating physics major.
%		\item[Sarah Nelson Scholarship] 2008-2009, GCSU\\
%			Presented to exceptional math majors.
%		\item[President's List] Fall 2008, Spring 2009\\
%			Awarded for 4.0 semester GPA 
%		\item[Dean's List] Sprint 2008, Fall 2009, Spring 2010, Spring 2007, Fall 2007\\
%			Awarded for 3.5 or higher semester GPA
%		\item[Georgia HOPE Scholarship] 2006-2010, University System of Georgia\\
%			Awarded for maintaining above 3.0 overall GPA
	\end{description}
\end{minipage}



\end{document}