\documentclass[11pt]{article}
\usepackage[colorlinks=true,urlcolor=blue]{hyperref}
\usepackage{multicol}

\usepackage{color}

\usepackage{amssymb}
\usepackage{amsmath}
\usepackage{amsfonts}

\usepackage{graphicx}

\usepackage[misc]{ifsym}

\setlength{\textwidth}{6.5in}
\setlength{\oddsidemargin}{0in}
\setlength{\evensidemargin}{0in}
\setlength{\topmargin}{-.5in}
\setlength{\headheight}{0in}
\setlength{\textheight}{9in}

\renewcommand*\ttdefault{txtt}
%\newcommand\0{\scalebox{-1}[1]{0}}
%\usepackage[T1]{fontenc}

\newcommand{\dotprod}{\boldsymbol{\cdot}}
\newcommand{\binary}[1]{\texttt{#1}}
\newcommand{\hex}[1]{\textsf{#1}}


\pagenumbering{gobble}

\begin{document}

\begin{center}
 {\bf S. Reece Boston}\\
 {\bf 501 NC 54 Apt E9, Carrboro, NC 27510}\\
 {\bf 770-355-0261}\\
 {\bf \href{mailto:rboston@ad.unc.edu}{rboston@ad.unc.edu}}
\end{center}

\section*{Education}
\begin{minipage}{\textwidth}
	\begin{description}
		\item[Ph.D., Physics, University of North Carolina-Chapel Hill, ongoing]
		\item[M.S., Physics, University of Georgia, 2015, GPA 3.88]
		\item[B.S., Mathematics and Physics, Georgia College, 2010, Cum Laude, GPA  3.81]
		\item[A.A., Mathematics, Georgia Perimeter College, 2007, GPA 3.33]
	\end{description}
\end{minipage}

\section*{Published Work}
	\begin{itemize}
		\item Boston, S.~Reece, Bart H.~Dunlap, J.~C.~Clemens, and Charles R.~Evans, ``The Limits of Newtonian White Dwarf Asteroseismology.'' [In Draft]\newline
		\underline{Role:} developed theory for post-newtonian pulsations, created C++ code for high-precision numerical analysis of stellar pulsations in newtonian, post-newtonian physics.
		\item de Souza, Rafael, \underline{S.~Reece Boston}, Alain Coc, and Christian Iliadis, ``Thermonuclear fusion rates for tritium+deuterium using Bayesian methods.''  \href{http://journals.aps.org/prc/abstract/10.1103/PhysRevC.99.014619}{Physical Review C}, (2018).\newline
		\underline{Role:} early analysis with Bayesian MCMC in R with JAGS, rewrote legacy fortran code into C++ to calculate S-factor for use with JAGS library for hundred-fold increase in productivity.
		\item Boston, S.~Reece, ``Time Travel in Transformation Optics.''  \href{http://journals.aps.org/prd/abstract/10.1103/PhysRevD.91.124035}{Physical Review D}, (2015).\newline
		\underline{Role:} Mathematical calculation of a metamaterial that mimics time-travel spacetimes from general relativity.
	\end{itemize}

\section*{Research Experience}
\begin{minipage}{\textwidth}
%\begin{multicols}{2}
	\begin{description}
		\item[Research Assistant] Fall 2016 - Present\\
		Department of Physics and Astronomy, University of North Carolina - Chapel Hill\\
		\underline{Research Advisor}: Charles R.~Evans\\
		\underline{Topic}: The numerical calculation of pulsation frequencies for white dwarf and other stellar objects in classical and general relativistic settings.  Calculations performed in C++.
	\end{description}
%\end{multicols}
\end{minipage}

\section*{Languages}
\begin{minipage}{\columnwidth}
\begin{multicols}{3}
	\begin{itemize}
		\item English (native)
		\item Spanish (conversational)
		\item C++ (7+ years)
		\item R (9+ years)
		\item UNIX (7+ years)
		\item \LaTeX (9+ years)
	\end{itemize}
\end{multicols}
\end{minipage}


\section*{Teaching Experience}
\begin{minipage}{\textwidth}
%\begin{multicols}{2}
	\begin{description}
		\item[Research Mentor] Summer 2020-Present\\
		Department of Physics and Astronomy, University of North Carolina - Chapel Hill \\
		Role: Acting mentor for REU/Senior Honor's Thesis in relativistic pulsation of neutron stars and white dwarfs.
		\item[Physics Instructor] Summer 2019, Summer 2020\\
			Department of Physics and Astronomy, University of North Carolina - Chapel Hill \\
			Course: Physics for Life Sciences\\
			Recorded many of \href{https://www.youtube.com/playlist?list=PLGw8-QpmEugz6IxX75R1kvd9NM_53pQXV}{the online lectures} during COVID-19 response (Lec~7-10,14,26-27).
		\item[Research Mentor] Fall 2018 - Summer 2019\\
		Department of Physics and Astronomy, University of North Carolina - Chapel Hill \\
			Role: Mentoring NCCMS high school student in guided research project on relativistic pulsation of neutron stars.  Student won \href{https://www.societyforscience.org/regeneron-sts/science-talent-search-2019/}{Regeneron STS 2019 Scholarship}.
		\item[Teaching Assistant] Fall 2016 - Ongoing\\
			Department of Physics and Astronomy, University of North Carolina - Chapel Hill\\
			Courses: Numerical Methods (LOI:~python), Electronics Lab, Physics for Life Sciences
		\item[STEM Teacher] Fall 2015 - Spring 2016\\
			Mount Pisgah Christian School\\
			Courses: AP Physics 1, High School Physics, Introductory Programming (LOI:~C++)\\
			Coach: FIRST Robotics Competition, FIRST Lego League
		\item[Teaching Assistant] Fall 2010 - Spring 2015\\
			Department of Physics and Astronomy, University of Georgia\\
			Courses: Physics Labs, Scale-Up Physics for Engineers
	\end{description}
%\end{multicols}
\end{minipage}

\section*{Awards and Honors}
\begin{minipage}{\textwidth}
%\begin{multicols}{2}
	\begin{description}
		\item[NC Space Grant] 2020, UNC\\
			Awarded through NASA for promising gradate student work related to NASA missions.
		\item[Outstanding Physics TA] 2018, UNC\\
			Awarded for performance as  teaching  assistance.  Included monetary award.
%		\columnbreak
		\item[Outstanding Physics Major] 2010, GCSU\\
			Presented to top graduating physics major.
		\item[Sarah Nelson Scholarship] 2008-2009, GCSU\\
			Presented to exceptional math majors.
	\end{description}
%\end{multicols}
\end{minipage}

\end{document}
